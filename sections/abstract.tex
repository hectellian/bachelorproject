\beforeabstract
\prefacesection{Abstract}

This paper explores the Fusee Gelee exploit, a significant hardware vulnerability in the Nintendo Switch. It begins with an overview of the Switch's security architecture, focusing on the Boot ROM and secure boot process. The exploit leverages an unchecked buffer in the Nvidia Tegra X1 chip's USB recovery mode, allowing attackers to execute arbitrary code and bypass security measures.

We discuss the discovery and technical details of the exploit, its broader implications for hardware security, and the challenges in addressing such vulnerabilities. The paper also evaluates Nintendo's mitigation efforts, including hardware revisions and software updates, highlighting their limitations. Finally, we propose alternative strategies, such as secure enclaves and continuous security audits, to enhance hardware security. These strategies, while effective, involve trade-offs in cost and complexity. The Fusee Gelee exploit underscores the importance of robust security measures in hardware design.

\prefacesection{Disclaimer}

The research and discussions presented in this thesis are intended solely for educational purposes. The case studies, including the examination of the "fusee-gelee" vulnerability within the Nintendo Switch console, are explored to contribute to the academic understanding of hardware security and side-channel resistances. Under no circumstances should the content of this thesis be used to engage in unlawful activities, including the hacking or modification of devices such as the Nintendo Switch. The author and academic institution do not condone unauthorized hacking, do not provide guidance for engaging in such activities, and are not liable for any actions taken by individuals who misuse the information provided.

\afterpreface\afterabstract